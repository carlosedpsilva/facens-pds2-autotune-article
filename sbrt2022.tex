%% ------------- Portuguese version ------------
\documentclass{sbrt}
\usepackage[english,brazil]{babel}
\usepackage[utf8]{inputenc}
\newtheorem{theorem}{Teorema}
%% ---------------------------------------------

%% If writing in English, remove the lines above
%% and uncomment the lines below

%% ------------- English version ---------------
%\documentclass[english]{sbrt}
%\usepackage[english]{babel}
%\usepackage[utf8]{inputenc}
%\newtheorem{theorem}{Theorem}
%% ---------------------------------------------

\begin{document}

\title{O impacto do autotune nos sinais de áudio da voz}

\author{Carlos Silva, Gabriel Silvério, Geovanne Bueno, Matheus Frate}
% \thanks{Nome1 Sobrenome1, Departamento1, Universidade1, Cidade1-UF1, e-mail: xxxxx@yyyyy.zzzzz.br; Nome2 Sobrenome2, Departamento2, Universidade2, Cidade2-UF2, e-mail: xxxxx@yyyyy.zzzzz.br. Este trabalho foi parcialmente financiado por XXXXXXX (XX/XXXXX-X).}


\maketitle

% \markboth{XL SIMPÓSIO BRASILEIRO DE TELECOMUNICA\c{C}\~{O}ES E PROCESSAMENTO DE SINAIS - SBrT 2022, 25--28 DE SETEMBRO DE 2022, STA. RITA DO SAPUCAÍ, MG}{}


%% If writing in English, remove both 'resumo' and 'chave'
%% ------------- Portuguese version ------------
\begin{resumo}
Este artigo tem com objtivo identificar a as alterações sofridas pelos sinais de audio da voz humana quando utilizado o software auto-tune para correções nas possíveis imprecisões e erros.
\end{resumo}
\begin{chave}
Auto-tune, sinais, audio.
\end{chave}
%% ---------------------------------------------


\begin{abstract}
This article aims to identify the changes suffered by the audio signals of the human voice when using the auto-tune software to correct possible inaccuracies and errors.

\end{abstract}
\begin{keywords}
Auto-tune signals, audio.
\end{keywords}


\section{Introdução}

O auto-tune está extensivamente entrelaçado na cena musical da atual geração de músicos \cite{ref1}. Com isso surge o questionamentos de como é feito o processo de modificação do tom da voz do cantor, que ocorre pela alteração da frequência da voz para o tom mais próximo do esperado. 
"Por exemplo, se a frequência da voz estiver em 435 Hz a voz está desafinada, pois não existe um tom nessa frequência, então ajusta-se para o tom mais próximo, que é Lá de 440 Hz" (JUNIOR, 2020). Nesse projeto o estudo vai se basear na analise dos dados obtidos após o tratamento do audio, utilizando a transformada de fourier.


\subsection{Sobre o SBrT}

O SBrT ocorre anualmente promovendo o encontro científico de maior relevância nacional da área de telecomunicações em que são discutidos temas de primordial importância para a evolução da pesquisa e do desenvolvimento deste setor.

\section{Figuras e Tabelas}
A Tabela \ref{tab:tabela} é apenas um exemplo \cite{ref2}.
\begin{table}[htb]
\caption{\label{tab:tabela}O \textit{caption} vem antes da tabela.}
\begin{center}
{\tt
\begin{tabular}{|c||c|c|c|}\hline
&title page&odd page&even page\\\hline\hline
onesided&leftTEXT&leftTEXT&leftTEXT\\\hline
twosided&leftTEXT&rightTEXT&leftTEXT\\\hline
\end{tabular}
}
\end{center}
\end{table}

A Figura \ref{fig:figura} é apenas um exemplo \cite{ref2}.

\begin{figure}[hbt]
\begin{center}
\setlength{\unitlength}{0.0105in}%
\begin{picture}(242,156)(73,660)
\put( 75,660){\framebox(240,150){}} \put(105,741){\vector( 0, 1){
66}} \put(105,675){\vector( 0, 1){ 57}} \put( 96,759){\vector( 1,
0){204}} \put(105,789){\line( 1, 0){ 90}} \put(195,789){\line(
2,-1){ 90}} \put(105,711){\line( 1, 0){ 60}} \put(165,711){\line(
5,-3){ 60}} \put(225,675){\line( 1, 0){ 72}} \put(
96,714){\vector( 1, 0){204}} \put(
99,720){\makebox(0,0)[rb]{\raisebox{0pt}[0pt][0pt]{a}}}
\put(291,747){\makebox(0,0)[lb]{\raisebox{0pt}[0pt][0pt]{ o}}}
\put(291,702){\makebox(0,0)[lb]{\raisebox{0pt}[0pt][0pt]{ o}}}
\put( 99,795){\makebox(0,0)[rb]{\raisebox{0pt}[0pt][0pt]{ $M$}}}
\end{picture}
\end{center}
\caption{\label{fig:figura}Esta figura é apenas um exemplo. O
\textit{caption} deve vir após a figura.}
\end{figure}

\section{Equações}

Este é um exemplo de como incluir uma equação no texto.

\begin{equation}\label{eq:exemplo}
    h(t)=\sum_{n=0}^{N-1} \alpha_n\delta(t-\tau_n),
\end{equation}
onde $\alpha_n$ é o $n$-ésimo...

Observe que (\ref{eq:exemplo}) é apenas um exemplo. Existem diferentes formas de incluir equações com múltiplas linhas, como
\begin{equation} \label{eq:exemplo_multiplo}
    \begin{array}{ccl}
        x^2 & = & bx+c\\
        y^2 & = &\beta y+\gamma=0.
    \end{array}
\end{equation}
Eq. (\ref{eq:exemplo_multiplo}) é apenas um exemplo.

\section{Conclusões}
A versão final do artigo aceito para publicação nos Anais do XL Simpósio Brasileiro de Telecomunicações e Processamento de Sinais deve ser enviada, em formato PDF, no máximo até o dia especificado na chamada de trabalhos. O formato do artigo deve ser A4, coluna-dupla, 10pt, lado-único, e possuir no máximo 5 páginas. O Resumo e o \emph{Abstract} devem ter no máximo 100 palavras cada um.

\section*{Agradecimentos}
A Coordenação Técnica do SBrT~2022 agradece as coordenações dos simpósios anteriores promovidos pela Sociedade Brasileira de Telecomunicações por disponibilizarem este exemplo.

\begin{thebibliography}{99}
\bibitem{ref1} J. Diaz, \textit{The Fate of Auto-Tune}. MIT, Outubro 2009.
\bibitem{ref2} M. C. B. Junior, \textit{IMPLEMENTAÇÃO DE ALGORITMOS DE CORREÇÃO
AUTOMÁTICA DE TOM}, UFES, 2020
1999.
\end{thebibliography}


\appendix
Inserir as informações referentes aos apêndices aqui.


\end{document}
