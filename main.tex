\documentclass{sbrt}
\usepackage[english,brazil]{babel}
\usepackage[utf8]{inputenc}
\usepackage{amsmath}
\usepackage{natbib}
\newtheorem{theorem}{Teorema}

%% ---------------------------------------------

\begin{document}

\title{Autotune: como funciona a harmonização da voz humana}

\author{Carlos E. P. Silva, Gabriel M. Silvério, Geovanne B. Lopes, Matheus B. Frate}

\maketitle

%% ---------------------------------------------

\begin{resumo}
O Autotune é um software que permite aos usuários alterar o tom de sua voz utilizando um algoritmo que analisa o sinal de áudio e, em seguida, ajusta o tom da voz para que ele esteja em harmonia com o tom desejado. Este artigo tem como objetivo explicar como funciona a harmonização da voz humana.
\end{resumo}
\begin{chave}
Autotune, algoritmo, análise de sinal, equalização, filtragem, tom
\end{chave}

%% ---------------------------------------------

\begin{abstract}
Autotune is a software that allows users to change the pitch of their voice using an algorithm that analyzes the audio signal and then adjusts the pitch of the voice to be in harmony with the desired pitch. The purpose of this article is to explain how voice harmony works.
\end{abstract}
\begin{keywords}
Autotune, algorithm, signal analysis, equalization, filtering, pitch
\end{keywords}

%% ---------------------------------------------

\section{Introdução}

O auto-tune é um software que usa um algoritmo para processar sinais de voz humana. Ele analisa o sinal de áudio e, em seguida, ajusta o tom da voz para que ele esteja em harmonia com o sinal. O auto-tune pode ser usado para ajustar o tom da voz para qualquer tom, mas é mais comumente usado para ajustar o tom da voz para o tom de um instrumento musical~\cite{browning2014auto}.

O auto-tune está extensivamente entrelaçado na cena musical da atual geração de músicos~\cite{diaz2009fate}. Com isso surge o questionamento: como o algoritmo do auto-tune funciona?

O primeiro passo para compreender o funcionamento do algoritmo é aprender o que é o processamento de sinais. Em termos gerais, é o processo de analisar e manipular um sinal para extrair informações úteis ou para alterar o sinal de alguma forma. Um sinal pode ser qualquer coisa, desde um som até um gráfico de um fenômeno físico. O processamento de sinais pode ser feito de forma analógica ou digital. O processamento de sinais analógicos é feito usando circuitos eletrônicos, enquanto o processamento de sinais digitais é feito usando algoritmos e computadores~\cite{prandoni2008signal}.

\begin{quote} ``Por exemplo, se a frequência da voz estiver em 435 Hz a voz está desafinada, pois não existe um tom nessa frequência, então ajusta-se para o tom mais próximo, que é Lá de 440 H``~\cite{deimplementaccao}. \end{quote}

Este trabalho busca entender o processo de harmonização da voz, utilizando um programa em Python para poder fazer a analise dos dados obtidos e e assim ser feito o tratamento necessário para harmonização do tom.

%% ---------------------------------------------

\section{Fundamentação}

\subsection{Notas Musicais}

Na notação musical temos sete notas, Dó, Ré, Mi, Fá, Sol, Lá e Si. Elas são chamadas de notas naturais, pois ainda existem varias escalas que podem usar essas mesmas notas mas com diferentes frequências. A frequência dessas notas podem ser vistas na tabela abaixo~\cite{moretti2003prototipo}:

\begin{table}[h]
\centering
\caption{Notas Musicais}
\vspace{-0.2cm}
\begin{tabular}{c|c}
Nota & Frequência \\
\hline
Dó &261,63 \\
Ré &293,66 \\
Mi &329,63 \\
Fá &349,23 \\
Sol &392 \\
Lá &440 \\
Si &493,88
\end{tabular}
\end{table}

\subsection{Transformada Rápida de Fourier}

A transformada rápida de Fourier (FFT), é um metodo que  converte um sinal em componentes espectrais individuais para fornecer a frequência do sinal.

Desenvolvido em 1965 por  J. W. Cooley (IBM) e J. W. Tukey
(Bell Labs), teve como objetivo principal otimizar a utilização da Transformada Discreta de Fourier (DTF), para que fosse executada de forma mais rápida e com um trabalho computacional reduzido \cite{martins2016analise}.

Enquanto a implementação do DTF possuiam um custo computacional de $O(N^2)$ com a implementação do FFT o custo foi reduzido para $O(N log_2 N)$. A FFT mudou o uso de polinômios trigonométricos interpolatórios, organizando de forma que o número de pontos usados pode ser fatorado em potências de dois, dessa forma, sua utilização é mais eficiente por ser mais rápida e por realizar o cálculo da DFT e também sua inversa \cite{reis2008implementaccao}.

%% ---------------------------------------------

\section{Desenvolvimento}

Lorem ipsum.

%% ---------------------------------------------

\section{Conclusões}

Lorem ipsum.

%% ---------------------------------------------

\bibliographystyle{unsrt}
\bibliography{refs}

%% ---------------------------------------------

\end{document}
