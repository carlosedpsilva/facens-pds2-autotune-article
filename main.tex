\documentclass{sbrt}
\usepackage[english,brazil]{babel}
\usepackage[utf8]{inputenc}
\usepackage{natbib}
\newtheorem{theorem}{Teorema}

%% ---------------------------------------------

\begin{document}

\title{Autotune: como funciona a alteração do tom da voz}

\author{Carlos E. P. Silva, Gabriel M. Silvério, Geovanne B. Lopes, Matheus B. Frate}

\maketitle

%% ---------------------------------------------

\begin{resumo}
Este artigo tem com objtivo identificar a as alterações sofridas pelos sinais de audio da voz humana quando utilizado o software auto-tune para correções nas possíveis imprecisões e erros.
\end{resumo}

\begin{chave}
Auto-tune, sinais, audio.
\end{chave}

%% ---------------------------------------------

\begin{abstract}
This article aims to identify the changes suffered by the audio signals of the human voice when using the auto-tune software to correct possible inaccuracies and errors.
\end{abstract}

\begin{keywords}
Auto-tune signals, audio.
\end{keywords}

%% ---------------------------------------------

\section{Introdução}

O auto-tune está extensivamente entrelaçado na cena musical da atual geração de músicos \cite{diaz2009fate}. Com isso surge o questionamentos de como é feito o processo de modificação do tom da voz do cantor, que ocorre pela alteração da frequência da voz para o tom mais próximo do esperado.
"Por exemplo, se a frequência da voz estiver em 435 Hz a voz está desafinada, pois não existe um tom nessa frequência, então ajusta-se para o tom mais próximo, que é Lá de 440 Hz" \cite{deimplementaccao}. Nesse projeto o estudo vai se basear na analise dos dados obtidos após o tratamento do audio, utilizando a transformada de fourier.

%% ---------------------------------------------

\section{Conclusões}

Lorem ipsum.

%% ---------------------------------------------

\bibliographystyle{plain}
\bibliography{refs}

%% ---------------------------------------------

\end{document}
