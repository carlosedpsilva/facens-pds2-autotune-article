\documentclass{sbrt}
\usepackage[english,brazil]{babel}
\usepackage[utf8]{inputenc}
\usepackage{natbib}
\newtheorem{theorem}{Teorema}

%% ---------------------------------------------

\begin{document}

\title{Autotune: como funciona a alteração do tom da voz}

\author{Carlos E. P. Silva, Gabriel M. Silvério, Geovanne B. Lopes, Matheus B. Frate}

\maketitle

%% ---------------------------------------------

\begin{resumo}
Autotune é um software que permite aos usuários alterar o tom da sua voz. O autotune usa um algoritmo para analisar o
sinal de áudio e, em seguida, ajusta o tom da voz para que ele esteja em harmonia com o sinal. Este artigo tem como
objetivo explicar como funciona autotune e como ele pode ser usado para processar sinais de voz humana.
\end{resumo}
\begin{chave}
Autotune, software, voz, tom, instrumentos musicais, algoritmo, harmonia
\end{chave}

%% ---------------------------------------------

\begin{abstract}
Autotune is a software that allows users to change the pitch of their voice. It was originally developed to adjust the
pitch of musical instruments, but can also be used to process human voice signals. Autotune uses an algorithm to analyze
the audio signal and then adjusts the pitch of the voice to be in harmony with the signal. The purpose of this article
is to explain how autotune works and how it can be used to process human voice signals.
\end{abstract}
\begin{keywords}
Autotune, software, voice, pitch, musical instruments, algorithm, harmony
\end{keywords}

%% ---------------------------------------------

\section{Introdução}

O auto-tune está extensivamente entrelaçado na cena musical da atual geração de músicos~\cite{diaz2009fate}. Com isso
surge o questionamento: como o algoritmo do auto-tune funciona?

O primeiro passo para compreender o funcionamento do algoritmo é aprender o que é o processamento de sinais. Em termos
gerais, o processamento de sinais é o processo de analisar e manipular um sinal para extrair informações úteis ou para
alterar o sinal de alguma forma. Um sinal pode ser qualquer coisa, desde um som até um gráfico de um fenômeno físico. O
processamento de sinais pode ser feito de forma analógica ou digital. O processamento de sinais analógicos é feito
usando circuitos eletrônicos, enquanto o processamento de sinais digitais é feito usando algoritmos e
computadores~\cite{prandoni2008signal}.

O auto-tune é um software que usa um algoritmo para processar sinais de voz humana. Ele analisa o sinal de áudio e, em
seguida, ajusta o tom da voz para que ele esteja em harmonia com o sinal. O auto-tune pode ser usado para ajustar o tom
da voz para qualquer tom, mas é mais comumente usado para ajustar o tom da voz para o tom de um instrumento
musical~\cite{browning2014auto}.

\begin{quote} ``Por exemplo, se a frequência da voz estiver em 435 Hz a voz está desafinada, pois não existe um tom
nessa frequência, então ajusta-se para o tom mais próximo, que é Lá de 440 H``~\cite{deimplementaccao}. \end{quote}

%% ---------------------------------------------

\section{Conclusões}

Lorem ipsum.

%% ---------------------------------------------

\bibliographystyle{unsrt}
\bibliography{refs}

%% ---------------------------------------------

\end{document}
